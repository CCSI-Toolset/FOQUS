\section{File Formats}
\label{ap:psuadefiles}

Most UQ capabilities within FOQUS rely on PSUADE. As such, different UQ
components require input files in PSUADE formats. CSV (comma-separated values) files are also compatible. 
The specific requirements are explained in Sections \ref{sec:uq_tutorial} and \ref{sec:ouu_overview}.

\subsection{PSUADE Full File Format}
The following is an example of the full PSUADE file format. Comments in red do not appear in the file and are only for instructional purposes.
 \small
\begin{Verbatim}[commandchars=\\\{\}]
PSUADE_IO (Note : inputs not true inputs if pdf ~=U) \scriptsize\color{red}Start data block
5 2 6                    \scriptsize\color{red}5 inputs, 2 outputs, and 6 samples
1 1                      \scriptsize\color{red}Sample index, run value (0 if sample point has not been calculated.)
 -9.5979899497487442e-01  \scriptsize\color{red}Value to the first input of sample 1
  1.0552763819095490e-01  \scriptsize\color{red}Value to the second input of sample 1
  2.1608040201005019e-01  \scriptsize\color{red}Value to the third input of sample 1
 -2.1608040201005019e-10  \scriptsize\color{red}Value to the fourth input of sample 1
 -2.5628140703517588e-01  \scriptsize\color{red}Value to the fifth input of sample 1
 -1.6979984061153328e+00  \scriptsize\color{red}Value to the first output of sample 1 (9.99e+34 if undefined)
 -7.8296928608517824e-01  \scriptsize\color{red}Value to the second output of sample 1 (9.99e+34 if undefined)
2 1                      \scriptsize\color{red}Sample point 2. Run value is true (outputs calculated).
 -9.5477386934673336e-02
  8.5427135678391997e-02
 -9.7989949748743721e-01
 -4.8743718592964824e-01
  3.5175879396984966e-02
  9.7708275149071300e-01
  8.6655187317087226e-02
3 1                      \scriptsize\color{red}Sample point 3. Run value is true (outputs calculated).
 -6.9849246231155782e-01 
 -5.9798994974874375e-01 
 -9.6984924623115576e-01
  2.5125628140703515e-02
  8.1909547738693478e-01
 -6.4229247835711212e-02
  2.8546752874255432e-01 
4 1                      \scriptsize\color{red}Sample point 4. Run value is true (outputs calculated).
  2.1608040201005019e-01
  7.2864321608040195e-01
  4.9748743718592969e-01
  5.6783919597989962e-01
  6.7839195979899491e-01
 -4.7115433927748318e-01
 -3.5869634004753126e-01 
5 1                      \scriptsize\color{red}Sample point 5. Run value is true (outputs calculated).
  5.6783919597989962e-01
  5.4773869346733672e-01
 -2.2613065326633164e-01
  3.8693467336683418e-01
 -1.7587939698492461e-01
  6.8926859881410230e-03
 -2.7551395275787588e-01
6 0                      \scriptsize\color{red}Sample point 6. Run value is false (outputs not calculated).
-7.2864321608040195e-01
 2.1608040201005019e-01
 8.3919597989949746e-01
 3.5175879396984966e-02
 2.3618090452261309e-01
 9.9999999999999997e+34  \scriptsize\color{red}Output not calculated.
 9.9999999999999997e+34  \scriptsize\color{red}Output not calculated. 
PSUADE_IO      \scriptsize\color{red}End data block
PSUADE         \scriptsize\color{red}Start informational block
INPUT          \scriptsize\color{red}Start input information block
   dimension = 5     \scriptsize\color{red}Number of inputs
   variable 1 A0  =  -1.00000e+00   1.00000e+00  \scriptsize\color{red}{Input name \& range}
   variable 2 A1  =  -1.00000e+00   1.00000e+00
   variable 3 A2  =  -1.00000e+00   1.00000e+00
   variable 4 A3  =  -1.00000e+00   1.00000e+00
   variable 5 A4  =  -1.00000e+00   1.00000e+00
END            \scriptsize\color{red}End input information block
OUTPUT         \scriptsize\color{red}Start output information block
dimension = 2        \scriptsize\color{red}Number of outputs
variable 1 Y1        \scriptsize\color{red}{Output name}
variable 2 Y2
END            \scriptsize\color{red}End output information block
METHOD         \scriptsize\color{red}Start sampling method information block
   sampling = LH     \scriptsize\color{red}Latin Hypercube sampling
   num_samples = 6   \scriptsize\color{red}Number of samples
END            \scriptsize\color{red}End sampling method block
APPLICATION    \scriptsize\color{red}Start application block
   driver = NONE     \scriptsize\color{red}Name of driver program for calculating outputs (NONE for no driver)
END            \scriptsize\color{red}End application block
ANALYSIS       \scriptsize\color{red}Start analysis method information block
   analyzer output_id  = 1
   analyzer rstype = MARS  \scriptsize\color{red}Default response surface type
   diagnostics 1
END            \scriptsize\color{red}End analysis method information block
END            \scriptsize\color{red}End information block
\end{Verbatim}
\normalsize
This file format is accepted when:
\begin{itemize}
\item{The user loads an existing ensemble by clicking the \bu{Load from
    File} button from the main UQ screen (Figure \ref{fig:uq_screen}).}
\item{The user creates a new ensemble by clicking the \bu{Add New} button
  from the main UQ screen (Figure \ref{fig:uq_screen}) and selecting
  the \bu{Load all samples from a single file} radio button in the user's selection
  of sample generation (Figure \ref{fig:uq_sim_loadsample}).}
\item{The user performs optimization under uncertainty from the main OUU screen
  (Figure \ref{fig:ouu_screen}) and selects the \bu{Load Model From File}
  radio button for the user's model; for this file, the user does not need to specify
  the first block (i.e., the PSUADE\_IO block).}
\end{itemize}

This file format is written when:
\begin{itemize}
\item{The user saves an existing ensemble by clicking the \bu{Save
    Selected} button from the main UQ screen (Figure \ref{fig:uq_screen}).}
\end{itemize}

\subsection{PSUADE Sample File Format}

The following is an example of the sample file format. Comments in red do NOT appear in the file and are only for instructional purposes.

\begin{Verbatim}[commandchars=\\\{\}]
PSUADE_BEGIN  \scriptsize\color{red}Start data block
5 2           \scriptsize\color{red}5 samples, 2 inputs
1 4.0 -1.0    \scriptsize\color{red}Sample index, input values for sample point 1
2 3.0 2.0     \scriptsize\color{red}Sample index, input values for sample point 2
3 5.0 1.0     \scriptsize\color{red}Sample index, input values for sample point 3
4 2.0 1.5     \scriptsize\color{red}Sample index, input values for sample point 4
5 3.0 3.0     \scriptsize\color{red}Sample index, input values for sample point 5
PSUADE_END    \scriptsize\color{red}End data block
\end{Verbatim}

This file format is accepted when:
\begin{itemize}
\item{The user creates a new ensemble by clicking the \bu{Add New} button
  from the main UQ screen (Figure \ref{fig:uq_screen}) and selecting
  the \bu{Load all samples from a single file} radio button in the user's selection
  of sample generation (Figure \ref{fig:uq_sim_loadsample}).}
\item{The user creates a new ensemble by clicking the \bu{Add New} button
  from the main UQ screen (Figure \ref{fig:uq_screen}) and selecting
  the \bu{Choose sampling scheme} radio button in the user's selection
  of sample generation (Figure \ref{fig:uq_sim_dist}); in the
  \bu{Distributions} tab, if the user designates an input variable's PDF to be of
  type ``Sample'', the ``Param 1'' field will generate a \bu{Select
  	File} button that prompts for the sample file representing the input's PDF.}
\item{Similar to above, when the user enters Expert Mode within the Analysis
  dialog;
  within Expert Mode (Figure \ref{fig:uqt_rsaeua}), the user can change the input distribution before
  performing response surface based analysis.} 
\item{The user performs optimization under uncertainty from the main OUU screen
  (Figure \ref{fig:ouu_screen});
  if any of the variables are designated as random variables, the \bu{UQ
    Setup} tab will be displayed and
    any prompt for loading existing sample (e.g., ``Load existing sample
    for Z3'' or ``Load existing sample for Z4'') will require this file
    format. (Currently, the \bu{UQ Setup} tab is missing from the Figure
    because no variables have been designated as random.)}
\end{itemize}

This file format is written when:
\begin{itemize}
\item{The user wants to save the results of inference by clicking  \bu{Save
    Posterior Input Samples to File} within Bayesian Inference (Figure
  \ref{fig:uq_inf}), which is accessible from the Analysis screen of UQ
  (Figure \ref{fig:uq_analysisW}).}
\end{itemize}

\subsection{CSV (Comma Separated Values) File Format}
The following is an example of the CSV file format. Comments in red do not appear in the file and are only for instructional purposes. CSV files can be easily generated using Excel and exporting in the .csv format

\begin{Verbatim}[commandchars=\\\{\}]
A0,A1,A2,A3,A4,Y1,Y2  \scriptsize\color{red}Input variable names, then output variable names (if any)
-0.959,0.105,0.216,-2.16e-10,-0.256,-1.698,-0.783 \scriptsize\color{red}Values for the 
\scriptsize\color{red}first sample (Output values are not required if not calculated)
-0.095,0.085,-0.980,-0.487,0.035,0.978,0.087 \scriptsize\color{red}Values for the second sample
-0.698,-0.598,-0.970,0.025,0.819,-0.064,0.285 
0.216,0.729,0.497,0.568,0.678,-0.471,-0.359 
0.568,0.548,-0.226,0.387,-0.176,6.89e-03,-0.276  
\end{Verbatim}

Variable names are specified in the first line, with input names and then output names. Output names can be specified, even if there is no data available for them yet. Data is only required for inputs. In addition, the variable names line is not required in those places where a PSUADE sample file is acceptable. 

This file format is accepted when:
\begin{itemize}
	\item{The user loads an existing ensemble by clicking the \bu{Load from
			File} button from the main UQ screen (Figure \ref{fig:uq_screen}). Variable names are required.}
	\item{The user creates a new ensemble by clicking the \bu{Add New} button
		from the main UQ screen (Figure \ref{fig:uq_screen}) and selecting
		the \bu{Load all samples from a single file} radio button in the user's selection
		of sample generation (Figure \ref{fig:uq_sim_loadsample}).}
	\item{The user creates a new ensemble by clicking the \bu{Add New} button
		from the main UQ screen (Figure \ref{fig:uq_screen}) and selecting
		the \bu{Choose sampling scheme} radio button in the user's selection
		of sample generation (Figure \ref{fig:uq_sim_dist}); in the
		\bu{Distributions} tab, if the user designates an input variable's PDF to be of
		type ``Sample'', the ``Param 1'' field will generate a \bu{Select
			File} button that prompts for the sample file representing the input's PDF.}
	\item{Similar to above, when the user enters Expert Mode within the Analysis
		dialog;
		within Expert Mode (Figure \ref{fig:uqt_rsaeua}), the user can change the input distribution before
		performing response surface based analysis.} 
	\item{The user performs optimization under uncertainty from the main OUU screen
		(Figure \ref{fig:ouu_screen});
		if any of the variables are designated as random variables, the \bu{UQ
			Setup} tab will be displayed and
		any prompt for loading existing sample (e.g., ``Load existing sample
		for Z3'' or ``Load existing sample for Z4'') will require this file
		format. (Currently, the \bu{UQ Setup} tab is missing from the Figure
		because no variables have been designated as random.)}
	\end{itemize}
