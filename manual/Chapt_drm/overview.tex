\section{D-RM (Dynamic Reduced Models) Builder}\label{sec.overview.drm}

The Dynamic Reduced Model (D-RM) Builder is a software tool used to generate data-driven dynamic reduced models from high-fidelity dynamic models consisting of differential and algebraic equations (DAEs).  DAE-based models are usually computationally intensive to solve, especially when stiff DAEs are involved.  For instance, the sorbent-based bubbling fluidized bed CO$_2$ adsorber-reactor model contains over 20,000 DAEs and very small time steps ($<$0.001 second) have to be used to solve the stiff equations in the rigorous model.  The D-RMs generated by the D-RM Builder enable much faster computation of the system responses (up to several orders of magnitude faster), which enable the development of an advanced process control framework and the integration of the dynamic models within a large-scale dynamic simulation.

\subsection{Motivating Example}

The solid-sorbent-based bubbling fluidized bed CO$_2$ adsorber-reactor for the post-combustion carbon capture is a good example to which the D-RM Builder tool can be applied.  The dynamics of the adsorber-reactor was modeled by a CCSI team using Aspen Custom Modeler (ACM).  The high-fidelity dynamic model of the twin-bed adsorber-reactor contains over 20,000 equations.  The feed streams include CO$_2$-containing flue gas, solid sorbent, and multiple cooling water streams.  Since some equations are very stiff, a minimum integration time step of 0.001 second has to be used.  As a result, the computer CPU time required to calculate the response of the system over a period of operating time is much longer than the real operating time, especially when there is a change in model inputs (step or ramp change).  The data-driven D-RM can be generated by fitting the system outputs in response to the changes of system inputs through certain plant identification models.  The response of the system to the input changes can be simulated by the ACM model.  The D-RM Builder is provided for a user to configure the input and output variables of interest, prepare a sequence of step changes of input variables, launch ACM simulations, generate a D-RM, and export the D-RM in a form of a MATLAB function file (.m file), which can be called by MATLAB to calculate the system response given the model inputs with a CPU execution time a few orders of magnitude shorter than that required by the ACM model.  The speedup in CPU time enables the implementation of advanced process control (APC) systems.

\subsection{Features List}

The D-RM Builder is an application embedded in the graphic user interface (GUI) of FOQUS.  Data-driven D-RMs can be generated by the D-RM Builder based on a set of high-fidelity model outputs in response to a sequence of input changes within a range of operating conditions.  The D-RM Builder can read the input and output variables pre-configured through SinterConfigGUI application, which is also a part of the FOQUS package.  Through the D-RM Builder GUI, a user can select a set of the pre-configured input variables, specify their lower and upper limits, and prepare a sequence of step changes or ramp changes based on the Latin Hypercube Sampling (LHS) method with desired durations of the step changes to excite the system in a range of frequencies.  The simulations of high-fidelity models, currently implemented for ACM only, can be launched directly in the D-RM Builder through ConsoleSinter, which is also a part of the FOQUS package.  The simulation results can be used to generate the D-RMs.  A separate set of step-change sequence can also be generated and its response be simulated by the ACM model and predicted by the D-RM for validation purpose.  The generated D-RMs and the user inputs for case setup and configuration can be saved in a text file in JSON format with extension ``drmb''.  Once a case file is saved, the user can copy the file to a different directory or to a different computer where the D-RM Builder is installed and open the file later to continue the D-RM building process or visualize the properties of the generated D-RMs.  A D-RM Builder project could also be a subproject of a FOQUS project and saved as a part of FOQUS session.  When a saved FOQUS session file is opened, the D-RM Builder subproject is also opened.

Two types of D-RMs are supported in the current version including the Decoupled A-B Net (DABNet) model \cite{Sentoni_1998}, and the Nonlinear Auto-regressive Moving Average (NARMA) model \cite{Narendra_1997}.  Different building options are provided for the user to choose from including input delays, two-pole Laugerre formulation of DABNet model, and optimization of model parameters.  Both D-RM model types require the training of artificial neural networks (ANNs).  Two ANN training methods, back propagation (BP) and interior point optimization (IPOPT), are provided.  The generated D-RM can be validated by performing another set of high-fidelity model simulations with a different input change sequence and comparing the response to that predicted by the generated D-RMs.  The response calculated by the high-fidelity model and that by the D-RM can be displayed through the D-RM Builder$\text{'}$s GUI.  The accuracy of the generated D-RM can be visualized by comparing the D-RM predictions to the corresponding ACM predictions, including the relative errors and the coefficient of determination R$^2$.  For the state-space based DABNet model, Uncertainty Quantification (UQ) analysis can be performed on the validation data using Unscented Kalman Filter (UKF), providing the covariance matrices of state-space and output variables.  The messages of the model building process including command sequence are displayed in the main text window of the D-RM Builder, which can be saved to a log file for future reference.  The internal parameters of the generated D-RM can be exported to a MATLAB script file, which can be run in MATLAB to initialize a D-RM object instantiated from a MATLAB class.  The results of the high-fidelity model simulations and D-RM predictions for both the training and the validation input sequences can also be exported to text files in comma-separated value (CSV) format, which can be opened by Microsoft® Excel®.

Along with the D-RM Builder application, several MATLAB files are also provided as part of the software tool.  These files define the MATLAB classes that can be used to create the D-RM objects in the MATLAB workspace given the D-RM files exported from the D-RM Builder.  The functions in the D-RM objects can be called to perform dynamic simulations.
