\section{Tutorial -- Surrogates with UQ Tools}
\label{tutorial.surrogate.uq}

For the purpose of this tutorial, we will use ACOSSO to demonstrate the use
of a surrogate within the UQ module. The steps are the same regardless of
the surrogate tool chosen.

To perform the UQ analysis, Python is required for use the ``User Regression'' response surface that will be used.  Before starting this tutorial, you will need to install Python 2.7.x (not Python 3). (See \href{https://www.python.org/downloads/}{\textcolor{blue}{https://www.python.org/downloads/}}). In addition, if *.py files have been re-associated with other executables (e.g. editors), please change the association back to python.exe. 


\begin{enumerate}
\item{Load a fresh session by clicking the \bu{Session} button from the Home window. Select \textbf{\underline{Open Session}} and then navigate to the ``examples/UQ''
  directory. Select ``Rosenbrock\_no\_vectors.foqus.'' This will load a
  session with a simple flowsheet containing a single node.}
\item{Click \bu{Settings} and ensure that (1) FOQUS Flowsheet Run Method is
  set to ``Local'', and that (2) proper paths are set for PSUADE and RScript.} 
\item{Train an ACOSSO surrogate of this node by clicking the
  \bu{Surrogates} button from the Home window.
\begin{enumerate}
\item{Click \bu{Add Samples} and select ``Use Flowsheet''. This will
  display the \bu{Simulation Ensemble Setup} dialog.}
\item{Within this dialog, ensure all variables are set to ``Variable''
  type in the \bu{Distributions} tab. In the \bu{Sampling scheme} tab,
  select ``Monte Carlo'' as your sampling scheme, set the number of
  samples to 100, and then click \bu{Generate Samples} to generate the
  set of input values. Click \bu{Done} to return to the Surrogates screen.}
\item{Once sample generation completes, click the \bu{Uncertainty} button from the Home window.}
\item{Click the \bu{Launch}	button to generate the samples.}
\item{Click the \bu{Surrogates} button from the Home window. The \bu{Data} tab of the Surrogates
  screen should now displays a \bu{Flowsheet Results} table that is populated with
  the values of the new input samples.}
\item{From the \bu{Variables} tab, select all of the checkboxes. (There should
  be six checkboxes for input variables and one checkbox for output variable.) 
  Here, you are defining the inputs and outputs for your surrogate function.}
\item{From the \bu{Method Settings} tab, note the name of the file next to
  ``FOQUS Model (for UQ)''. This will be the name of the UQ driver file
  that contains the Python code that implements the surrogate function.}
\item{On top of this screen, select ``ACOSSO'' as your surrogate tool from the \textbf{\underline{Tool}} drop-down list and then click on the green arrow to start training the surrogate.}
\item{Once complete, a popup window will display, reminding you of the
  location of the drive file. Note the location as you will need this
  information later inside the UQ module.} 
\end{enumerate}
}
\item{Perform a response-surface-based uncertainty analysis by clicking the 
  \bu{Uncertainty} button from the Home window.
\begin{enumerate}
\item{In the \bu{Uncertainty Quantification Simulation Ensembles} table. A row corresponding to the ensemble that was just
  generated for surrogate training should be displayed. This same ensemble can be used or
  a new one can be created to be used as the test data set for analysis. 
  In the row corresponding to the ensemble
  to be analyzed, click the \bu{Analyze} button to proceed. This
  action will bring up an analysis dialog.} 
\item{Within this analysis dialog, navigate to ``Analysis'' section. For
  Step 1, select ``Response Surface''. For Step 3, select ``User
  Regression'' in the first drop-down list. Lastly, for ``User Regression File'',
  browse to the same location as the UQ driver file that was
  generated within the Surrogates module. (This is the same location that
  was previously noted from the popup message.) 
  At this point, your surrogate function is now set up as a user-defined
  response surface and all response-surface-based UQ analyses are accessible.}
\item{Click \bu{Validate} (Step 4) to perform response surface validation. Once
  complete, a figure with cross-validation results will be displayed: 
  a histogram of errors to the left and a plot of predicted values versus
  actual values to the right. For more information, refer to the UQ
  Tutorial in Section \ref{tutorial.uq.rs}.}
\item{Once a ``Response Surface'' has been validated, other UQ analysis options
  are available. Choose ``Uncertainty Analysis'' in Step 5 and click
  \bu{Analyze} to perform uncertainty analysis using your ACOSSO surrogate.}
\end{enumerate}
}
\end{enumerate}

During validation, if the error, ``RSAnalyzer: RSTest\_hs.m does
not exist.'' displays, this is likely caused by incompatibility with the surrogate
and the test data. An example scenario might be your test data has six
inputs, but your surrogate assumes five inputs. This is easily fixed by
returning to the Surrogates screen, clicking on the \bu{Variables}
tab, and making sure the appropriate selections are made (i.e., check off six
inputs instead of just five).
