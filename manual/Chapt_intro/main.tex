\chapter{Introduction}

The Framework for Optimization, Quantification of Uncertainty and Surrogates (FOQUS) software provides a graphical interface and standard platform for several Carbon Capture Simulation Initiative (CCSI) tools. The primary feature of FOQUS is its ability to interact with commonly-used chemical engineering process modeling software. Models constructed using a variety of software can be combined into a larger composite model. CCSI tools SimSinter and the Turbine Science Gateway (TSG) provide connectivity to external process simulation software. SimSinter provides a standard library to enable interfacing with other software; TSG provides a simulation job queuing system that can be used on: (1) a single workstation, (2) networked workstations, (3) cluster, or (4) cloud computing resources.

In FOQUS, simulations can be connected in a meta-flowsheet, which enables parts of a process to be modeled using the most appropriate software and combines them into a single large model, possibly including recycle streams. For example, in studying a carbon capture system for a coal-fired power plant: a power plant may be modeled in Thermoflex; a solvent-based carbon capture system may be modeled in Aspen Plus; and a compression system may be modeled in gPROMS. To optimize the entire system, these models can be combined into a single large model. The resulting meta-flowsheet can be used for simulation-based optimization, uncertainty quantification (UQ), or generation of surrogate models. Information extracted from the simulations can also be used to construct a heat integration optimization problem to determine the best way to use excess heat in the process. Heat integration can be combined with simulation-based optimization.

This section provides brief overview and motivating examples, for different uses of FOQUS.  

\section{Simulation Based Optimization}

Simulation-based optimization considers a process simulation to be a black box model, which is a model where the mathematical details are not known. In this case, models are evaluated using process simulation software; multiple models can be combined to form larger models. Due to the long run times and the limitations of the methods used, a limited set of optimization variables (usually less than 30) is considered. Simulation-based optimization has some advantages and disadvantages, compared to equation-based optimization methods. With simulation-based optimization, there is no need to provide simplified algebraic models, problem formulation is relatively simple, and a good solution can usually be obtained; however, a provably-global optimum cannot be found and it is impractical to deal with very large numbers of variables. Large numbers of variables may be found in superstructure and heat integration problems where the structure of a process is being optimized. Both simulation and equation-based optimization methods are used in CCSI.

Capture of CO$_2$ from a pulverized coal-fired power plant involves several very different systems including: a boiler, steam cycle, flue gas desulfurization, carbon capture, and CO$_2$ compression. It is convenient to separate many of these processes into smaller, more reliable simulations. The different processes may also be better simulated in different software packages.  Although some process simulation software contains optimization features, there are several reasons these may not be practical for a large composite system. It may be hard to develop a large model of the entire system that reliably converges. Many optimization methods have a difficult time dealing with simulation errors, and many black box derivative free optimization solvers are better able to handle occasional simulation failures. It may not be practical to simulate the entire process accurately using a single tool. Derivatives are also difficult to estimate for many systems when models do not provide exact derivatives, making derivative-free methods a good option.

The motivating example used to demonstrate the optimization framework is fairly simple. The system consists of a series of bubbling fluidized bed (BFB) CO$_2$ adsorbers and regenerators modeled in Aspen Custom Modeler (ACM). The details of the BFB system are described in the CCSI BFB model documentation. A cost analysis for a 650 MW power plant and capture system is presented in an Excel spreadsheet. The simulation and spreadsheet files are provided in the examples directory in the FOQUS installation directory (see the tutorial in Section \ref{tutorial.sim.flowsheet} for more information). The spreadsheet contains capital cost as well as operating and maintenance cost estimates, which are used to estimate the cost of electricity.  

In this example, the objective function is the cost of electricity; the decision variables are design and operating variables in the ACM model. The cost of electricity is minimized while maintaining a 90 CO$_2$ percent capture rate. The BFB system model and the cost of electricity are contained in separate models connected in a FOQUS flowsheet, which enables the cost of electricity to be calculated in Excel, using data acquired from the ACM model. See Sections \ref{tutorial.sim.flowsheet} and \ref{sec.opt.tutorial} for more information about the optimization problem.

\section{Uncertainty Quantification}
\label{sec:uq_intro}

The Uncertainty Quantification (UQ) module of FOQUS encompasses a rich selection of mathematical,
statistical, and diagnostic tools for application users to perform UQ
studies on their simulation models. The PSUADE tool provides most of the
UQ functionality available in FOQUS \citep{Tong_2011}. The recommended
systematic multi-step approach consists of the following steps:
\begin{enumerate}
	\item Define the objectives of the analysis (e.g., identify the most
     important sources of uncertainties).
	\item Specify a simulation model to be studied. Acquire the model input
     files and the executable that runs the simulation (i.e., an executable
     that uses the specified inputs and generates model outputs). Identify
     the outputs of interest, identify all relevant sources of
     uncertainties, and ensure that they can be used as input variables to
     the simulation model.
	\item Select some or all input parameters that have uncertainty
     attributed. Characterize the prior probability distribution of these
     selected parameters by specifying the upper/lower bounds. For
     non-uniform prior distributions (e.g., Gaussian), additional
     information (e.g., mean and standard deviation) is required to define
     the shape of the prior distribution. This prior distribution
     represents the user's best initial guess about the selected
     parameters' uncertainties.
	\item Identify, if available, relevant data from physical experiments
     that can be used for model parameter calibration. Model calibration is
     a process that applies the observational data to update the prior
     distribution. The model calibration correlates the observational data to predict a distribution as a result. 
	\item Select a sample scheme and sample size. From this information, a
     set of input values are sampled from the prior distribution. The
     choice of sampling scheme (which affects how the samples populate the
     input space) depends on the UQ objective(s) specified in Step 1.
	\item ``Run'' the input samples. Running the input samples is the
     process where each sampled input value is fed to the simulation
     executable (specified in Step 2) and the corresponding output value is
     returned.
	\item Analyze the results and make decisions on how to proceed.
\end{enumerate}

Steps 1-4 are often done through expert knowledge elicitation and/or
literature search. Steps 5-7 can be achieved through software provided
in the FOQUS UQ module. 

The FOQUS UQ module provides a number of sampling and analysis methods, including:
\begin{itemize}
	\item Parameter screening methods: computes the importance of input
     parameters to identify which are important (to be kept in subsequent
     analyses) and which to ignore (to be weeded out).
	\item Response surface (used interchangeably with `surrogate')
     construction: approximates the relationship between the input samples
     and their outputs via a smooth mathematical function. This response
     surface or surrogate can then be used in place of the actual
     simulation model to speed up lengthy simulations.
	\item Response surface validation methods: evaluates how well a given
     response surface fits the data. This is important for choosing
     different response surfaces.
	\item Basic uncertainty analysis: propagates input uncertainty to output
     uncertainty.
	\item Sensitivity analysis methods: quantifies how much varying an input
     value can impact the resulting output value. 
	\item Bayesian calibration: applies observational data to refine the
     estimate of input uncertainties.
	\item Visualization tools: views computed distributions and response
     surfaces.
	\item Diagnostics tools (mainly in the form of scatter plots):
     checks samples and model behaviors (e.g., outliers).
\end{itemize}

The adsorber 650.1 subsystem process model is used to demonstrate the UQ
framework. The A650.1 process model was developed and is continuously
refined by our Process Synthesis and Design Team. The model is based on
their design and optimization of an initial full-scale design of a solid
sorbent capture system for a net 650 MW (before capture) supercritical
pulverized coal power plant. The A650.1 model describes a solid
sorbent-based carbon capture system that uses the NETL-32D
sorbent. NETL-32D is a mixture of polyethyleneamine (PEI) and aminosilanes
impregnated into the mesoporous structure of a silica substrate. CO$_2$
removal is achieved through chemical reactions between the amine sites
within the sorbent. The A650.1 model is implemented in Aspen Custom Modeler
(ACM) and contains many components (e.g., adsorbers, regenerators,
compressors, heat exchangers). For the UQ analyses, this manual focuses is on the adsorber units, which are responsible for the adsorption of
CO$_2$ from the input flue gas. 

In its original form, the A650.1 model consists of a deterministic simulation model, which means to consider all the parameters (e.g. chemical reaction parameters, heat and mass transfer coefficients) to have a fixed value (most likely fixed to a mean value, lower or upper bound for robustness). With the FOQUS UQ module, the model uncertainties can be addressed. Thus, UQ analysis of the A650.1 model would help to develop a robust design by addressing the following questions:  
\begin{itemize}
	\item How accurately does each subsystem model predict actual system performance (under uncertain operating conditions)? 
	\item Which input parameters should be examined to improve prediction accuracy? 
	\item What is each input parameters' contribution to prediction uncertainty?
\end{itemize}

\section{Optimization Under Uncertainty}

The Optimization Under Uncertainty (OUU) module in FOQUS is an extension of simulation-based optimization by including the contribution of model
parameter uncertainties in the objective function. OUU is useful when inclusion of uncertainties may significantly alter the 
optimal design configurations. A straightforward approach to include the effect of uncertainty is to replace the objective 
function with its statistical mean on an ensemble drawn from the probability distributions of the continuous uncertain 
parameters (other options are available in FOQUS). Alternatively, users can provide a set of `scenarios', where each scenario is
associated with a probability. The latter case is often called `scenario optimization.' The FOQUS OUU accommodates both 
continuous and scenario-based uncertain parameters. OUU makes use of the flowsheet for evaluations of the objective function.
Naturally, OUU requires more computational resources than deterministic optimization. However, the ensemble runs can be
launched in parallel so ideally, the turnaround time remains about the same as that of deterministic optimization if
high performance computing capability (such as the CCSI Turbine gateway) is used in conjunction with FOQUS.

\section{Heat Integration}

The Heat Integration tool maximizes heat utilization within the entire process. For example, in a carbon capture process, the regenerator needs to be heated by steam while the adsorber (or absorber) needs to be cooled by water. Heat integration aims to match all available heat sources and sinks within the process so that heating or cooling requirements can be satisfied internally as much as possible without using utilities. In carbon capture, heat integration not only reduces steam consumption for the regenerator, but also recovers part of heat from the adsorber to be used in other parts of the process. The heat integration tool is tightly integrated with process simulation and optimization software. The graphical interface transfers the relevant information from process simulation results to heat integration inputs. The heat integration tool consists on a mathematical model solved using mathematical programming to minimize utility usage. The General Algebraic Modeling System (GAMS) software is used for this purpose. Finally, the graphical interface sends the heat integration results to other simulation inputs or optimization inputs. Capital cost calculation for the heat exchanger network is not considered in the current version of heat integration tool due to its complexity.

There are many heat integration opportunities in a power plant with steam cycles, carbon capture processes and CO$_2$ compressors. For example, potential heat sources include adsorbers in the capture process and intercoolers between CO$_2$ compressors. Potential cold sources include feed water heaters in the steam cycle and regenerators in the capture process. Note: In a power plant, steam is usually extracted from the steam cycle instead of purchasing it from outside; therefore, heat integration not only reduces utility cost but also increases the net power output and efficiency. The motivating example consists of a BFB CO$_2$ adsorber/regenerator process model (in ACM), a multi-stage CO$_2$ compression process model (in ACM), steam cycle calculations (in Python), and heat integration calculations (in GAMS). Heat integration returns minimum utility cost, minimum hot and cold utility consumption, and minimum number of heat exchangers. These results can be used as final outputs or inputs for steam cycle calculations. Correlations for net power outputs to steam extraction or heat addition, which are obtained from the Thermoflex model, are used in steam cycle calculations. Steam cycle calculations provide the net power output and efficiency with carbon capture and sequestration (CCS), and possibly heat integration. This example demonstrates the net efficiency of a power plant with CCS, which can be (potentially) significantly increased after using the heat integration tool.

\section{Surrogate Models}

Process simulations are often time consuming and occasionally fail to converge. For mathematical optimization, it is sometimes necessary to replace a simulation with a surrogate model, which is a simplified model that executes much faster. FOQUS contains tools for creating and quantifying the uncertainty associated with surrogate models.

\subsection{ALAMO}
While simulation based optimization, can often do a good job of providing optimal design and operating conditions for a predetermined flowsheet, it cannot provide an optimal flowsheet.  To obtain a more optimal flowsheet, a mixed integer nonlinear program must be solved. These types of problems cannot generally be solved using simulation based optimization. A solution is to generate relatively simple algebraic models that accurately represent the high fidelity models. FOQUS currently provides an interface for ALAMO \citep{Cozad_2014}, which builds surrogate model that are well suited for superstructure optimization.

\subsection{ACOSSO}
The Adaptive Component Selection and Shrinkage Operator (ACOSSO) surface approximation was developed under the Smoothing Spline Analysis of Variance (SS-ANOVA) modeling framework \citep{Storlie_2011}. As it is a smoothing type method, ACOSSO works best when the underlying function is somewhat smooth. For functions which are known to have sharp changes or peaks, etc., other methods may be more appropriate. Since it implicitly performs variable selection, ACOSSO can also work well when there are a large number of input variables. To facilitate the description of ACOSSO, the univariate smoothing spline is reviewed first. The ACOSSO procedure also allows for categorical inputs \citep{Storlie_2013}.

\subsection{BSS-ANOVA}
The Bayesian Smoothing Spline ANOVA (BSS-ANOVA) is essentially a Bayesian version of ACOSSO \citep{Reich_2009}. It is Gaussian Process (GP) model with a non-conventional covariance function that borrows its form from SS-ANOVA. It tackles the high dimensionality (of inputs) on two fronts: (1) variable selection to eliminate uninformative variables from the model and (2) restricting the level of interactions involved among the variables in the model. This is done through a fully Bayesian approach which can also allow for categorical input variables with relative ease. Since it is closely related to ACOSSO, it generally works well in similar settings as ACOSSO. The BSS-ANOVA procedure also allows for categorical inputs \citep{Storlie_2013}.

\subsection{iREVEAL}

The iREVEAL surrogate model builds Kriging-based surrogate models. The purpose of iREVEAL is to provide surrogate models for CFD and other high-fidelity model simulations. CFD models are generally very computationally expensive and, therefore, difficult to combine with simpler process models. The iREVEAL surrogate models can be seamlessly embedded to an ACM and Aspen Plus flowsheet and, therefore, used by FOQUS.

\section{Dynamic Reduced-Order Model Builder (D-RM Builder)}

While a meta-flowsheet can be developed in FOQUS to model and optimize a system in steady-state operating conditions, the dynamics of individual components and the entire system also need to be modeled. The best design of a coal-fired power generation and CO$_2$ capture system should not only achieve the lowest cost per kilowatt-hour of electricity generated in the optimal steady-state operating conditions, but also ensure easy startup and shutdown, quick and smooth transition for load change, and process noise rejection.  Advanced process control (APC) methods require the development of dynamic models of the system and its individual components. Typically the dynamics of a process can by modeled by a set of differential and algebraic equations (DAEs) based on the first principles. A DAE-based high-fidelity dynamic model is simulated by a DAE solver such as ACM. Due to the complexity of the process involved in CO$_2$ capture, such as the one inside a BFB adsorber, the Central Processing Unit (CPU) time required to solve the dynamics of the process is too long to implement the APC method. Therefore, a dynamic reduced-order model (D-RM) is desired with its execution time reduced by several orders of magnitude.  The D-RM Builder, as a part of FOQUS software package, can be used to generate data-driven D-RMs. With dynamic variables configured by SinterConfigGUI, the D-RM Builder enables a user to import a dynamic ACM model, sample the input space to form a sequence of step changes, launch the high-fidelity ACM simulations, generate a D-RM, and finally visualize and validate the D-RM.




