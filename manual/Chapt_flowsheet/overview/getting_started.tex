\section{Starting FOQUS}\label{sec.flowsheet.starting.foqus}

The Windows FOQUS installer will add a FOQUS menu under ``CCSI Tools'' in the Windows start menu. Upon software installation (see the FOQUS bundle installation manual), the user can start the graphical interface using one of the following methods:
\begin{enumerate}
	\item	Select FOQUS from the Windows start menu. Note: FOQUS may be slow starting the first time, which oftentimes is related to installed anti-virus software.
	\item  Double-click a file with the ``.foqus'' extension in the Windows file explorer. FOQUS will automatically load the selected file. 
	\item	The easiest way to launch FOQUS from the command line, if it has been installed from the Windows installer, is to open ``Turbine Console'' from the FOQUS menu in the Windows start menu. The following command can be used in the command prompt:
	\begin{itemize}
		\item \texttt{foqus}
		\item \texttt{foqus --load <path to session file>}
	\end{itemize}
	The load option can take a relative or absolute path. If a relative path is specified, it is relative to the directory where the command is executed.
\end{enumerate}

The first time FOQUS is started, the user is prompted to specify a working directory. The working directory preference is stored in ``\%APPDATA\%\textbackslash.foqus.cfg'' on Windows (APPDATA is an environment variable). On Linux or OSX, the working directory is specified in ``\$HOME/.foqus.cfg''. Additionally the user can override the working directory when starting FOQUS by using the ``\texttt{--}working\_dir $\langle$working dir$\rangle$'' or ``-w $\langle$working dir$\rangle$'' command line option. Log files, user plugins, and files related to other FOQUS tools are stored in the working directory. The working directory can be changed at a later time from within FOQUS. A full list of FOQUS command line arguments is available using the ``-h'' or ``\texttt{--}help'' arguments.
