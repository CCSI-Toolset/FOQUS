\chapter{Debugging}

This chapter contains information that may be helpful in resolving a problem or filing a bug report.

\section{How to Debug}
Log files may contain very useful information when reporting problems. The log files are contained in the logs sub-directory of the FOQUS working directory. To change the log message levels in FOQUS go to the FOQUS \textbf{\underline{Settings}} button from the Home window. From there various log settings can be changed. The debugging log level provides the highest level of information.

Almost any error that occurs in FOQUS should be logged. Occasionally, an error may occur that is difficult to find, or causes FOQUS to crash before logging it. In that case the ``FOQUS Console'' application can be used.  All output from FOQUS, including messages that cannot be seen otherwise will be shown in a ``cmd'' window which will remain open even after FOQUS closes.

When running heat integration, the debugging information can be found in \\
\textbackslash gams\textbackslash HeatIntegration.lst. This file includes detailed results and errors returned by GAMS.

Most UQ routines interact with PSUADE via Python wrappers. When PSUADE is running, the stdout is written to psuadelog in the working directory. (At present, only some PSUADE commands write to this log; however, this will be standardized in the near future so that all PSUADE commands write to this log.) Other errors that are due to the Python wrappers or PySide GUI components are written to the logs subdirectory in the working directory.

\section{Known Issues}
The following are known unresolved issues:
\begin{itemize}
	\item The FOQUS flowsheet can be edited while a flowsheet evaluation, optimization, or UQ is running. This should not be allowed, and may cause problems.
	\item With the windows installer, FOQUS may produce output to standard error, especially if it immediately fails to launch. Output is usually caught and redirected to the FOQUS log and displayed in dialog boxes within FOQUS, but rare instances may occur where error messages are not caught.  Output to standard error is logged in the directory with foqus.exe in the file foqus.exe.log.  The user does not typically have permission to write to the FOQUS install location, so an error message such as ``Cannot write to foqus.exe.log'' will be displayed.  If this occurs there are two solutions (1) change the permissions for the FOQUS install directory or (2) run ``FOQUS Console'' application, which will direct FOQUS output to the ``cmd'' window.
	\item The win32com module generates Python code, which it needs to run.  This code is generated in the FOQUS install location ``\textbackslash dist\textbackslash win32com\textbackslash gen\_py.''  In some cases there may be a problem writing to that directory due to permission settings. This will prevent FOQUS from running simulations locally. If this error is encountered the solution is to make the ``gen\_py'' directory user writable.  So far, in testing, this error seems to occur in Windows 8 and 10, but not 7.
	\item The user regression analysis features, iREVEAL and
          ALAMO, of the UQ tool requires a separate Python 2.7
          installation. Furthermore, Python must be both in PATH
          variable and associated with .py files.  Details on
          installing Python and fixing any issues encountered may
          be found in the iREVEAL Installation Guide and the iREVEAL
          User Manual, Known Issues section.
        \item FOQUS has trouble getting files from Turbine and saving them to the DMF when dealing with files in Turbine involving directories.
        \item The default port for TurbineLite is 8080. If another program is already using port 8000, there will be an error in FOQUS when connecting to TurbineLite.  In the \bu{Turbine} Tab of the Settings window, there is a tool to change the TurbineLite port.  If the TurbineLite port is changed the configuration file that FOQUS uses to connect to TurbineLite, must also be changed.
\end{itemize}

\section{Reporting Issues}
To report an issue, please send an email to:\\
\textcolor{blue}{\uline{\href{mailto://ccsi-support@acceleratecarboncapture.org}{ccsi-support@acceleratecarboncapture.org}}}

Please include detailed steps on how to reproduce the error, including screenshots and log files.
